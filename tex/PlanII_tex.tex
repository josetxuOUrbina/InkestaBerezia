\documentclass[12pt,a4paper]{article}
\usepackage[utf8]{}
\usepackage[T1]{fontenc}
\title{Révision français}
\author{Louis Hardy}
\date{Mai 2019}

\renewcommand{\baselinestretch}{0.2}

\setlength{\hoffset}{-18pt}         
\setlength{\oddsidemargin}{0pt} % Marge gauche sur pages impaires
\setlength{\evensidemargin}{0pt} % Marge gauche sur pages paires
\setlength{\marginparwidth}{54pt} % Largeur de note dans la marge
\setlength{\textwidth}{481pt} % Largeur de la zone de texte (17cm)
\setlength{\voffset}{-18pt} % Bon pour DOS
\setlength{\marginparsep}{7pt} % Séparation de la marge
\setlength{\topmargin}{0pt} % Pas de marge en haut
\setlength{\headheight}{13pt} % Haut de page
\setlength{\headsep}{4pt} % Entre le haut de page et le texte
\setlength{\footskip}{27pt} % Bas de page + séparation
\setlength{\textheight}{720pt} % Hauteur de la zone de texte (25cm)

\usepackage{natbib}
\usepackage{graphicx}

\begin{document}

\section*{La princesse de clève}
	(1678) Une esthétique classique
\begin{enumerate}
   \item Une incarnation de la perfection
   \begin{itemize}
     \item Le portrait physique
     \item Fine fleur de l'aristocratie
     \item L'importance du portrait moral
   \end{itemize}
   \item Une éducation irréprochable
   \begin{itemize}
       \item Éloge de Mme de Chartres chargée de l'éducation de sa fille
       \item Préparation à la vie de cour et aux risques qu'elle comporte
       \item Une vision exigeante de l'amour et de l'honnête femme
    \end{itemize}
\end{enumerate}
\noindent\hrulefill

\section*{La princesse de clève}
	(1678) Une esthétique classique
\begin{enumerate}
   \item Une incarnation de la perfection
   \begin{itemize}
     \item Le portrait physique
     \item Fine fleur de l'aristocratie
     \item L'importance du portrait moral
   \end{itemize}
   \item Une éducation irréprochable
   \begin{itemize}
       \item Éloge de Mme de Chartres chargée de l'éducation de sa fille
       \item Préparation à la vie de cour et aux risques qu'elle comporte
       \item Une vision exigeante de l'amour et de l'honnête femme
    \end{itemize}
\end{enumerate}
\noindent\hrulefill

\section*{La princesse de clève}
	(1678) Une esthétique classique
\begin{enumerate}
   \item Une incarnation de la perfection
   \begin{itemize}
     \item Le portrait physique
     \item Fine fleur de l'aristocratie
     \item L'importance du portrait moral
   \end{itemize}
   \item Une éducation irréprochable
   \begin{itemize}
       \item Éloge de Mme de Chartres chargée de l'éducation de sa fille
       \item Préparation à la vie de cour et aux risques qu'elle comporte
       \item Une vision exigeante de l'amour et de l'honnête femme
    \end{itemize}
\end{enumerate}

\section*{La princesse de clève}
	(1678) Une esthétique classique
\begin{enumerate}
   \item Une incarnation de la perfection
   \begin{itemize}
     \item Le portrait physique
     \item Fine fleur de l'aristocratie
     \item L'importance du portrait moral
   \end{itemize}
   \item Une éducation irréprochable
   \begin{itemize}
       \item Éloge de Mme de Chartres chargée de l'éducation de sa fille
       \item Préparation à la vie de cour et aux risques qu'elle comporte
       \item Une vision exigeante de l'amour et de l'honnête femme
    \end{itemize}
\end{enumerate}
\noindent\hrulefill

\section*{La princesse de clève}
	(1678) Une esthétique classique
\begin{enumerate}
   \item Une incarnation de la perfection
   \begin{itemize}
     \item Le portrait physique
     \item Fine fleur de l'aristocratie
     \item L'importance du portrait moral
   \end{itemize}
   \item Une éducation irréprochable
   \begin{itemize}
       \item Éloge de Mme de Chartres chargée de l'éducation de sa fille
       \item Préparation à la vie de cour et aux risques qu'elle comporte
       \item Une vision exigeante de l'amour et de l'honnête femme
    \end{itemize}
\end{enumerate}
\noindent\hrulefill

\section*{La princesse de clève}
	(1678) Une esthétique classique
\begin{enumerate}
   \item Une incarnation de la perfection
   \begin{itemize}
     \item Le portrait physique
     \item Fine fleur de l'aristocratie
     \item L'importance du portrait moral
   \end{itemize}
   \item Une éducation irréprochable
   \begin{itemize}
       \item Éloge de Mme de Chartres chargée de l'éducation de sa fille
       \item Préparation à la vie de cour et aux risques qu'elle comporte
       \item Une vision exigeante de l'amour et de l'honnête femme
    \end{itemize}
\end{enumerate}

\section*{La princesse de clève}
	(1678) Une esthétique classique
\begin{enumerate}
   \item Une incarnation de la perfection
   \begin{itemize}
     \item Le portrait physique
     \item Fine fleur de l'aristocratie
     \item L'importance du portrait moral
   \end{itemize}
   \item Une éducation irréprochable
   \begin{itemize}
       \item Éloge de Mme de Chartres chargée de l'éducation de sa fille
       \item Préparation à la vie de cour et aux risques qu'elle comporte
       \item Une vision exigeante de l'amour et de l'honnête femme
    \end{itemize}
\end{enumerate}
\noindent\hrulefill

\section*{La princesse de clève}
	(1678) Une esthétique classique
\begin{enumerate}
   \item Une incarnation de la perfection
   \begin{itemize}
     \item Le portrait physique
     \item Fine fleur de l'aristocratie
     \item L'importance du portrait moral
   \end{itemize}
   \item Une éducation irréprochable
   \begin{itemize}
       \item Éloge de Mme de Chartres chargée de l'éducation de sa fille
       \item Préparation à la vie de cour et aux risques qu'elle comporte
       \item Une vision exigeante de l'amour et de l'honnête femme
    \end{itemize}
\end{enumerate}
\noindent\hrulefill

\section*{La princesse de clève}
	(1678) Une esthétique classique
\begin{enumerate}
   \item Une incarnation de la perfection
   \begin{itemize}
     \item Le portrait physique
     \item Fine fleur de l'aristocratie
     \item L'importance du portrait moral
   \end{itemize}
   \item Une éducation irréprochable
   \begin{itemize}
       \item Éloge de Mme de Chartres chargée de l'éducation de sa fille
       \item Préparation à la vie de cour et aux risques qu'elle comporte
       \item Une vision exigeante de l'amour et de l'honnête femme
    \end{itemize}
\end{enumerate}

\end{document}
